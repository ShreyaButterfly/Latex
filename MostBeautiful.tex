\documentclass[10pt]{beamer}
\usefonttheme{professionalfonts}
\usepackage[utf8]{inputenc}
\usepackage[T1]{fontenc}
\usepackage{lmodern}
\usepackage[english]{babel}
\usepackage{amsmath}
\usepackage{amsfonts}
\usepackage{amssymb}
\usepackage{graphicx}
\usepackage{mathtools}
\usetheme{AnnArbor}
%\usetheme{UniKlu}

\newcommand{\iu}{{i\mkern1mu}}

\begin{document}



\author{\textcolor{orange}{Shreya Balaji}}
\title{Exploring a Beautiful Mathematical Identity}
\subtitle{\huge\(e^{\iu \pi} +1 = 0\)}
%\logo{"index.png"}
\institute{ Berkeley Carroll School}
\date{May 6, 2020}
%\subject{Hi}
%\setbeamercovered{transparent}
%\setbeamertemplate{navigation symbols}{}
\titlegraphic{\includegraphics[width=2cm]{index.jpeg}}


\begin{frame}[plain]
	\maketitle
	%\date{05/06/2020}
\end{frame}

\begin{frame}
	\frametitle{Why is the Identity Beautiful \hspace{25pt} \textcolor{pink}{\Huge\(e^{\iu \pi} + 1 =0\)}}
	
	Euler's identity is beautiful because it combines five of the most important numbers in mathematics into one equation. These are \dots
	\begin{list}{$\Xi$}{}
	\item 1 - the first positive integer
	\item 0 - the concept of nothingness \{together with 1, it makes the basis of binary numbers\}
	\item {\bf${\pi}$} - the ratio of a circle's circumference to its diameter. An irrational number.
	\item $e$ - the base of natural logarithms also known as Euler's number
	\item $\iu$ - the "imaginary" square root of -1, also known as an imaginary number.
	\end{list}
\vspace{20pt}
Over the next 10-15 minutes, I will attempt to show you two different proofs for the equation.
	
\end{frame}

\begin{frame}
	\frametitle{ History of Mathematicians Involved \hspace{25pt} \textcolor{pink}{\Huge\(e^{\iu \pi} +1 = 0\)}}

	A number o\label{key}f mathematicians laid the building blocks for understanding this equation.  As you can see, it has taken painstaking work of many mathematicians for us to come to such an elegant equation. Below are are some of the mathematicians whose works contributed to this equation :
	\vspace{10pt}

	\begin{list}{$\int$}{}
		\item Jacob Bernoulli - Swiss mathematician invented $e$
		\item  Gottfried Wilhelm Leibniz  - Along with Newton invented Calculus
		\item Issac Newton - Along with Leibniz invented Calculus
		\item Carl Friedrich Gauss - One of the greatest mathematicians of all time - Imaginary numbers $\iu$
		\item Leonhard Euler - Worked on imaginary numbers
		\item Augustin-Louis Cauchy - Worked on imaginary numbers
		\item Greek, Chinese and Indian Mathematicians - Contributed to $\pi$
		\item James Gregory, Brook Taylor and Colin Maclaurin - Contributed to understanding of expansion series
	\end{list}

\end{frame}

\begin{frame}
	\frametitle{The concept of exponential function $e^x$ \hspace{25pt} \textcolor{pink}{\Huge\(e^{\iu \pi} +1 = 0\)}}
	
The number $e$ (sometimes called the natural number) is Euler's number. It is an important mathematical constant that is equal to approximately 2.718. When it is used as the base of a logarithm, the corresponding logarithm is called the natural logarithm, written as $\ln(x)$. \vspace{10pt}

Most of the definitions for $e$ involves calculus. One definition of $e$ will be seen later in this presentation. There are several important uses of $e$, one of which is the calculation of continuous compounding of interest in the field of finance. \vspace{10pt}

Jacob Bernoulli, the founder of the concept $e$ discovered this constant by asking many questions about the amount of money in a bank account after continous compounding of interest. He finally came up with the formula: \\

$$e^x = \lim_{n\to\infty}\Big(1+\dfrac{x}{n}\Big)^n$$
	
\end{frame}
\begin{frame}
	\frametitle{Taylor and Maclaurin Series \hspace{25pt} \textcolor{pink}{\Huge\(e^{\iu \pi} +1 = 0\)}}
	James Gregory and Brook Taylor invented the idea of infinite series expansion of functions.  Scottish mathematician Colin Maclaurin used the work extensively to study functions centered around zero.  The basic idea of Maclaurin series is that any function can be expanded into a infinite series centered around zero as given below :
	{\large {$$f(x) = f(0)+x.\dfrac{f'(0)}{1!}+ x^2.\dfrac{f''(0)}{2!} +x^3.\dfrac{f'''(0)}{3!} + \dots$$}}
	You can re-center it to any point $a$ on the x axis by shifting the equation to  :
	{\large {$$f(x) = f(a)+(x-a).\dfrac{f'(a)}{1!}+ (x-a)^2.\dfrac{f''(a)}{2!} +(x-a)^3.\dfrac{f'''(a)}{3!} + \dots$$}}
	
	But for this presentation we will stick to centering around zero. 
	\vspace{10pt}

	In the next three slides, we will see the application of the same for three basic functions: $e(x) \dots \sin(x) \dots \cos(x)$
\end{frame}

\begin{frame}
	\frametitle{Taylor Series for e(x) \hspace{25pt} \textcolor{pink}{\Huge\(e^{\iu \pi} +1 = 0\)}}
	\begin{center}
		\begin{tabular}{|c|c|c|c|c|c|}
			\hline
			Order &$f(x)$ & $f'(x)$ & $f''(x)$ & $f'''(x)$ & $f''''(x)$ \\
			\hline
			Function & $e^x$  & $e^x$   & $e^x$    & $e^x$     & $e^x$      \\
			\hline
			Value at 0 & 1      & 1       & 1        & 1         & 1          \\
			\hline
		\end{tabular}
	\end{center}
	\vspace{20pt}
	Applying the Maclaurin formula we see that the infinite series expansion for $e^x$ is :

	{\large $$f(x) = f(0)+x.\dfrac{f'(0)}{1!}+ x^2.\dfrac{f''(0)}{2!} +x^3.\dfrac{f'''(0)}{3!} + \dots$$
	\large $$e^x = 1+x.\dfrac{1}{1!}+ x^2.\dfrac{1}{2!} +x^3.\dfrac{1}{3!} + \dots$$}

	By extension:
	{\large $$e^{{\iu}.x} = 1+\iu.x.\dfrac{1}{1!}- x^2.\dfrac{1}{2!} -\iu.x^3.\dfrac{1}{3!} + x^4.\dfrac{1}{4!} \dots$$}
\end{frame}

\begin{frame}
	\frametitle{Taylor Series for $\sin(x)$ \hspace{25pt} \textcolor{pink}{\Huge\(e^{\iu \pi} +1 = 0\)}}
	\begin{center}
		\begin{tabular}{|c|c|c|c|c|c|}
			\hline
			Order & \(f(x)\)    & \(f'(x)\)   & \(f''(x)\)   & \(f'''(x)\)  & \(f''''(x)\) \\
			\hline
			Function & \(\sin(x)\) & \(\cos(x)\) & \(-\sin(x)\) & \(-\cos(x)\) & \(\sin(x)\)  \\
			\hline
			Value at 0 & 0           & 1           & 0            & -1           & 0            \\
			\hline
		\end{tabular}
	\end{center}
	\vspace{20pt}
	Applying the Maclaurin formula we see that the infinite series expansion for $\sin(x)$ is :
	{\large $$f(x) = f(0)+x.\dfrac{f'(0)}{1!}+ x^2.\dfrac{f''(0)}{2!} +x^3.\dfrac{f'''(0)}{3!} + \dots$$
	\large $$\sin(x) = 0+x.\dfrac{1}{1!} -x^3.\dfrac{1}{3!} + x^5.\dfrac{1}{5!} +\dots$$}	
	By extension: 
	{\large $$\iu.\sin(x) = 0+\iu.x.\dfrac{1}{1!} -\iu.x^3.\dfrac{1}{3!} + \iu.x^5.\dfrac{1}{5!} +\dots$$}

\end{frame}

\begin{frame}
	\frametitle{Taylor Series for $\cos(x)$ \hspace{25pt} \textcolor{pink}{\Huge\(e^{\iu \pi} +1 = 0\)}}
	\begin{center}
		\begin{tabular}{|c|c|c|c|c|c|}
			\hline
			Order & $f(x)$    & $f'(x)$    & $f''(x)$  & $f'''(x)$ & $f''''(x)$ \\
			\hline
			Function & $\cos(x)$ & $-\sin(x)$ & $-cos(x)$ & $\sin(x)$ & $\cos(x)$  \\
			\hline
			Value at 0 & 1         & 0          & -1        & 0         & 1          \\
			\hline
		\end{tabular}
	\end{center}
	\vspace{20pt}
	Applying the Maclaurin formula we see that the infinite series expansion for $\cos(x)$ is :

	\large $$f(x) = f(0)+x.\dfrac{f'(0)}{1!}+ x^2.\dfrac{f''(0)}{2!} +x^3.\dfrac{f'''(0)}{3!} + \dots$$

	\large $$\cos(x) = 1- x^2.\dfrac{1}{2!} +x^4.\dfrac{1}{4!} + \dots$$

\end{frame}

\begin{frame}
	\frametitle{Bringing it all together \hspace{25pt} \textcolor{pink}{\Huge\(e^{\iu \pi} +1 = 0\)}}
	\begin{center}
	$\cos(x) + \iu.\sin(x) = 1+\iu.x.\dfrac{1}{1!}- x^2.\dfrac{1}{2!} -\iu.x^3.\dfrac{1}{3!} +x^4.\dfrac{1}{4!} + \iu.x^5.\dfrac{1}{5!} +\dots = e^{\iu.x}$
	\end{center}

	From the equations on the previous slides, we can see that \dots
	\begin{center}
		$\cos(x) + \iu.\sin(x) = e^{\iu.x}$ 
	\end{center} 
	If $x = \pi$ \dots
	\begin{center}
		$\cos(\pi) + \iu.\sin(\pi) = e^{\iu.\pi}$ \\
	\end{center}
	Then \dots
	\begin{center}
		$e^{\iu.\pi} + 1 = 0$
	\end{center}
\end{frame}




\begin{frame}
	\frametitle{Quick review of Complex Numbers \hspace{25pt} \textcolor{pink}{\Huge\(e^{\iu \pi} +1 = 0\)}}
	A complex number as a real part and an imaginary part.  Lets assume two complex numbers :
	\begin{list}{$\odot$}{}
		\item  $z = a+\iu b$
		\item  $w = c+\iu d$
	\end{list}
Then the following rules apply to complex numbers
	\begin{list}{$\odot$}{}
		\item $ z+w = (a+c) + \iu(b+d) $\\
		
		\item $ z-w = (a-c) + \iu(b-d) $ \\
		
		\item $ z*w = (ac-bd) +\iu(bc+ad)$
		
		\item $z^2 = (a^2-b^2) +\iu(2ab)$ \\
		
		\item magnitude of $|z| = r = \sqrt{a^2 +b^2}$ 
		
		\item The trignometric representation is $z = r(cos(\theta)+\iu\sin(\theta))$ where $\theta = tan^{-1}(\frac{b}{a})$
		
	\end{list}
\end{frame}

\begin{frame}
	\frametitle{A Graphical Proof \hspace{25pt} \textcolor{pink}{\Huge\(e^{\iu \pi} +1  = 0\)}}
	%\begin{center}
	Now I will show you all a graphical representation of this beautiful mathematical formula on Geogebra. The graphical proof of the equation is visually pleasing. We will now switch to Geogebra to go over the second way of proving the equation.
	%\end{center}
\end{frame}


\begin{frame}
	\begin{center}
		\Huge \textcolor{blue}{Thank You} \\
		\vspace{25pt}
		\huge \textcolor{pink}{Don't be a : ${\dfrac{d^3x}{dt^3}}$}
	\end{center}
\end{frame}

\end{document}
